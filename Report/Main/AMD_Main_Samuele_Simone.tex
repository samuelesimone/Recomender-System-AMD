% !TeX spellcheck = en_US 
\documentclass[12pt,english]{report}
\usepackage{tesi}
% CORSO DI LAUREA:
\def\myCDL{Master degree in\\Computer Science}

% TITOLO REPORT:
\def\myTitle{Algorithms for massive datasets \\
\large{Final report about Recommender System}}

% AUTORE:
\def\myName{Samuele Simone}
\def\myMat{Matr. Nr. 11910A}

\def\myRefereeA{Prof. Dario Malchiodi}

% ANNO ACCADEMICO
\def\myYY{2022-2023}

% Il seguente comando introduce un elenco delle figure dopo l'indice (facoltativo)
%\figurespagetrue

% Il seguente comando introduce un elenco delle tabelle dopo l'indice (facoltativo)
%\tablespagetrue


% Package di formato
\usepackage[a4paper]{geometry}		% Formato del foglio
\usepackage[english]{babel}			% Supporto per l'italiano
\usepackage[utf8]{inputenc}			% Supporto per UTF-8
%\usepackage[a-1b]{pdfx}			% File conforme allo standard PDF-A (obbligatorio per la consegna)

% Package per la grafica
\usepackage{graphicx}				% Funzioni avanzate per le immagini
\usepackage{hologo}					% Bibtex logo with \hologo{BibTeX}
%\usepackage{epsfig}				% Permette immagini in EPS
%\usepackage{xcolor}				% Gestione avanzata dei colori

% Package tipografici
\usepackage{amssymb,amsmath,amsthm} % Simboli matematici
\usepackage{listings}				% Scrittura di codice

% Package ipertesto
\usepackage{url}					% Visualizza e rendere interattii gli URL
\usepackage{hyperref}				% Rende interattivi i collegamenti interni


\begin{document}
% Creazione automatica del frontespizio
\frontespizio
{\raggedleft \large \sl \textit{I/We declare that this material, which I/We now submit for assessment, is entirely my/our own work and has not been taken from the work of others, save and to the extent that such work has been cited and acknowledged within the text of my/our work. I/We understand that plagiarism, collusion, and copying are grave and serious offences in the university and accept the penalties that would be imposed should I engage in plagiarism, collusion or copying. This assignment, or any part of it, has not been previously submitted by me/us or any other person for assessment on this or any other course of study.} \\}


\afterpreface

This report is made up of 8 chapters. In the \textbf{Chapter~\ref{ch:introduction}} we give at the reader a general view of what is a recommender system and where we can find it.
\textbf{Chapter~\ref{ch:environment}}, we start to discuss about the development setup for the project. Then, in the \textbf{Chapter~\ref{ch:dataset}} we focus on the dataset, how is composed and we explore the data. Later, in the \textbf{Chapter~\ref{ch:recsys}} we explain the recommender system. the different approaches and the mechanisms behind. In order to evaluate the project developed there are some notion about scalability and complexity in the \textbf{Chapter~\ref{ch:scalability}}. Consequently , we summarize the aspects and the results obtained during the various experiments in the \textbf{Chapter~\ref{ch:results}}. Finally there is the conclusion in last \textbf{Chapter~\ref{ch:conclusion}}.

\chapter{Introduction}\label{ch:introduction}
A recommender system is used everywhere nowadays. Indeed all big companies are pushing in these systems because they can increase the sells about their product, e.g., when we are scrolling a product on Amazon, then they show a list of recommendation based on the item selected.\par

\chapter{Environment setup}\label{ch:environment}
\chapter{Dataset: A look inside}\label{ch:dataset}
\chapter{Recommender system}\label{ch:recsys}
\chapter{Creation of a Content-based Recommender system}\label{ch:recsyscontbased}
\chapter{Scalability and complexity}\label{ch:scalability}
\chapter{Results and experiments}\label{ch:results}
\chapter{Conclusion}\label{ch:conclusion}




\end{document}
